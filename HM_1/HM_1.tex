\documentclass[a4paper,12pt]{article} % добавить leqno в [] для нумерации слева

%%% Работа с русским языком
\usepackage{cmap}					% поиск в PDF
\usepackage{mathtext} 				% русские буквы в фомулах
\usepackage[T2A]{fontenc}			% кодировка
\usepackage[utf8]{inputenc}			% кодировка исходного текста
\usepackage[english,russian]{babel}	% локализация и переносы
\usepackage[left=1cm,right=2cm,top=2cm,bottom=2cm,footskip=1cm,includefoot]
{geometry}
%%% Дополнительная работа с математикой
\usepackage{amsmath,amsfonts,amssymb,amsthm,mathtools} % AMS
\usepackage{icomma} % "Умная" запятая: $0,2$ --- число, $0, 2$ --- перечисление

%% Номера формул
\mathtoolsset{showonlyrefs=true} % Показывать номера только у тех формул, на которые есть \eqref{} в тексте.

%% Перенос знаков в формулах (по Львовскому)
\newcommand*{\hm}[1]{#1\nobreak\discretionary{}
	{\hbox{$\mathsurround=0pt #1$}}{}}

%%% Работа с картинками
\usepackage{graphicx}  % Для вставки рисунков
\graphicspath{{images/}{images2/}}  % папки с картинками
\setlength\fboxsep{3pt} % Отступ рамки \fbox{} от рисунка
\setlength\fboxrule{1pt} % Толщина линий рамки \fbox{}
\usepackage{wrapfig} % Обтекание рисунков и таблиц текстом

%%% Работа с таблицами
\usepackage{array,tabularx,tabulary,booktabs} % Дополнительная работа с таблицами

\usepackage{longtable}  % Длинные таблицы
\usepackage{multirow} % Слияние строк в таблице
\newcommand{\RomanNumeralCaps}[1]
{\MakeUppercase{\romannumeral #1}}

\usepackage{pgfplots, pgfplotstable}
\pgfplotsset{compat=1.9}
\usepackage{cancel}
\usepackage{circuitikz}

\begin{document} % конец преамбулы, начало документа

\hfill
\begin{minipage}{4.5cm}
	Просвирин Кирилл\\
	712 группа\\
\end{minipage}

\begin{center}
	\Large\textbf{Линейная алгебра\\} 
	\large Задание 1. Неделя 1
\end{center}

\noindent\textbf{15.45(2).}
Вычислить:
\begin{align}\label{key}
	&\left(
		\begin{array}{ccc|ccc}
		2 & -1 & 0    & 1 & 0 & 0\\
		0 & 2 & -1    & 0 & 1 & 0\\
		-1 & -1 & 1   & 0 & 0 & 1
		\end{array}
	\right)
	\sim
	\left(
		\begin{array}{ccc|ccc}
		2 & -1 & 0    & 1 & 0 & 0\\
		0 & 2 & -1    & 0 & 1 & 0\\
		0 & -3 & 2    & 1 & 0 & 2
		\end{array}
	\right)
	\sim
	\left(
		\begin{array}{ccc|ccc}
		2 & -1 & 0    & 1 & 0 & 0\\
		0 & 2 & -1    & 0 & 1 & 0\\
		0 & 0 & 1     & 2 & 3 & 4
		\end{array}
	\right)
	\sim\\
	&\sim\left(
		\begin{array}{ccc|ccc}
		2 & -1 & 0    & 1 & 0 & 0\\
		0 & 2 & 0    & 2 & 4 & 4\\
		0 & 0 & 1     & 2 & 3 & 4
		\end{array}
	\right)
	\sim
	\left(
		\begin{array}{ccc|ccc}
		4 & 0 & 0    & 4 & 4 & 4\\
		0 & 1 & 0    & 1 & 2 & 2\\
		0 & 0 & 1     & 2 & 3 & 4
		\end{array}
	\right)
	\sim
	\left(
		\begin{array}{ccc|ccc}
		1 & 0 & 0    & 1 & 1 & 1\\
		0 & 1 & 0    & 1 & 2 & 2\\
		0 & 0 & 1     & 2 & 3 & 4
		\end{array}
	\right)~~~\Longrightarrow\\
	&\textbf{Ответ: }
	\left(
	\begin{array}{ccc}
	2 & -1 & 0\\
	0 & 2 & -1\\
	-1 & -1 & 1
	\end{array}
	\right)^{-1}
	=
	\left(
		\begin{array}{ccc}
		1 & 1 & 1\\
		1 & 2 & 2\\
		2 & 3 & 4
		\end{array}
	\right)
\end{align}

\noindent\textbf{15.54(3).}
Вычислить:
\begin{align*}
	&\left(
		\begin{array}{ccccc|ccccc}
		1&1&\cdots&1&1					&1&0&\cdots&0&0\\
		0&1&\cdots&1&1    				&0&1&\cdots&0&0\\
		\vdots&\vdots&\ddots&\vdots&\vdots&\vdots
		&\vdots&\ddots&\vdots&\vdots\\
		0&0&\cdots&1&1     				&0&0&\cdots&1&0\\
		0&0&\cdots&0&1 					&0&0&\cdots&0&1\\
		\end{array}
	\right)
	\stackrel{(1)-(2)}{\sim}
	\left(
		\begin{array}{ccccc|ccccc}
		1&0&\cdots&0&0					&1&-1&\cdots&0&0\\
		0&1&\cdots&1&1    				&0&1&\cdots&0&0\\
		\vdots&\vdots&\ddots&\vdots&\vdots&\vdots
		&\vdots&\ddots&\vdots&\vdots\\
		0&0&\cdots&1&1     				&0&0&\cdots&1&0\\
		0&0&\cdots&0&1 					&0&0&\cdots&0&1\\
		\end{array}
	\right)
	\stackrel{(2)-(1)}{\stackrel{(3)-(4)}{\stackrel{\cdots}{\stackrel{(n-1)-(n)}{\sim}}}}\\
	&\sim\left(
	\begin{array}{ccccc|ccccc}
	1&0&\cdots&0&0					&1&-1&\cdots&0&0\\
	0&1&\cdots&0&0    				&0&1&\cdots&0&0\\
	\vdots&\vdots&\ddots&\vdots&\vdots&\vdots
	&\vdots&\ddots&\vdots&\vdots\\
	0&0&\cdots&1&0     				&0&0&\cdots&1&-1\\
	0&0&\cdots&0&1 					&0&0&\cdots&0&1\\
	\end{array}
	\right)
\end{align*}

\noindent\textbf{15.54(12*).}
Вычислить:
\begin{align*}\small
	\left(
		\begin{array}{ccccccc|ccccccc}
		1&1&0&\cdots&0&0&0					&1&0&0&\cdots&0&0&0\\
		0&1&1&\cdots&0&0&0    				&0&1&0&\cdots&0&0&0\\
		0&0&1&\cdots&0&0&0					&0&0&1&\cdots&0&0&0\\
		\vdots&\vdots&\vdots&\ddots&\vdots&\vdots&\vdots
		&\vdots&\vdots&\vdots&\ddots&\vdots&\vdots&\vdots\\
		0&0&0&\cdots&1&1&0     				&0&0&0&\cdots&1&0&0\\
		0&0&0&\cdots&0&1&1					&0&0&0&\cdots&0&1&0\\
		0&0&0&\cdots&0&0&1					&0&0&0&\cdots&0&0&1\\
		\end{array}
	\right)
	\stackrel{(n-1)-(n)}{\sim}
	\left(
	\begin{array}{ccccccc|ccccccc}
	1&1&0&\cdots&0&0&0					&1&0&0&\cdots&0&0&0\\
	0&1&1&\cdots&0&0&0    				&0&1&0&\cdots&0&0&0\\
	0&0&1&\cdots&0&0&0					&0&0&1&\cdots&0&0&0\\
	\vdots&\vdots&\vdots&\ddots&\vdots&\vdots&\vdots
	&\vdots&\vdots&\vdots&\ddots&\vdots&\vdots&\vdots\\
	0&0&0&\cdots&1&1&0     				&0&0&0&\cdots&1&0&0\\
	0&0&0&\cdots&0&1&0					&0&0&0&\cdots&0&1&-1\\
	0&0&0&\cdots&0&0&1					&0&0&0&\cdots&0&0&1\\
	\end{array}
	\right)
\end{align*}

$\small
	\stackrel{(n-2)-(n-1)}{\stackrel{(n-3)-(n-4)}{\stackrel{\cdots}{\stackrel{(1)-(2)}{\sim}}}}
	\left(
	\begin{array}{ccccc|ccccc}
	1&1&0&\cdots&0					&1&-1&1&\cdots&(-1)^{n-1}\\
	0&1&1&\cdots&0    				&0&1&-1&\cdots&(-1)^{n-2}\\
	0&0&1&\cdots&0					&0&0&1&\cdots&(-1)^{n-3}\\
	\vdots&\vdots&\vdots&\ddots&\vdots&\vdots&\vdots
	&\vdots&\ddots&\vdots\\
	0&0&0&\cdots&1     				&0&0&0&\cdots&1\\
	\end{array}
	\right)
$\\\\

\noindent\textbf{15.56.}
Пусть $ A^m=O $. Доказать, что 
$ (E-A)^{-1}=E+A+\cdots+A^{m-1}$.\\

\noindent $ \square $ По определению $ A^{-1} $~--- обратная к 
$ A \iff AA^{-1}=A^{-1}A=E $. Докажем оба равенства.
\begin{enumerate}
	\item $ (E-A)(E+A+\cdots+A^{m-1})\stackrel{?}{=}E $.
	Раскрывая скобки получим
	\[
		E+A+\cdots+A^{m-1}-A-\cdots-A^{m-1}=E \Longrightarrow \textbf{верно}
	\]
	\item $ (E+A+\cdots+A^{m-1})(E-A)\stackrel{?}{=}E $.
	Раскрывая скобки получим
	\[
	E-\underbrace{A+A}-\underbrace{A^2+A^2}+\cdots+
	\underbrace{A^{m-1}-A^{m-1}}-0=E \Longrightarrow \textbf{верно}
	\]
\end{enumerate}\hfill$ \blacksquare $

\noindent\textbf{15.58.} Проверить формулу $ (S^{-1}AS)^m=S^{-1}A^mS. $


\[
	(S^{-1}AS)^m=S^{-1}A\cancel{(S\cdot S^{-1})}AS\cdot\ldots\cdot S^{-1}AS=
	S^{-1}\underbrace{A\cdot\ldots\cdot A}_{m~раз}S=S^{-1}A^mS.
\]



\end{document}



